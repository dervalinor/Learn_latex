\documentclass{article}
\usepackage[utf8]{inputenc}
\usepackage{graphicx}
\graphicspath{ {Pictures/} }

\title{H-bonding catalysis mechanism of aromatic electrophilic sub
substitution between phenol and formaldehyde}
\author{Miguel Angel Riascos }
\date{June 2022}

\begin{document}
\maketitle

\section{Introduction}

\textbf{Phenolic resins} are first the polymeric product produced commercially \cite{phenolresins} and widely use as moulding plastics, coating, industrial bonding and ablative materials. \footnote {Ablative materials consist of a composite of polymeric materials such as an epoxy filled with either phenolic microballoons or cork \cite{ablative}.}



\begin{figure}[h]
    \centering
 	\includegraphics[width=0.25\textwidth]{Phenolic_resins_structure.png}
    \caption{Phenol resins structure.}
    \label{fig:mesh1}
\end{figure}

\begin{thebibliography}{9}
\bibitem{ablative}
J. Halchak (2001) \emph{Encyclopedia of Materials: Science and Technology},  Addison-Wesley Professional.

\bibitem{phenolresins}
J. A. Brydson (1999) \emph{in Plastics Materials (Seventh Edition)} https://www.sciencedirect.com/topics/chemical-engineering/phenolic-resins.
\end{thebibliography}

\end{document}