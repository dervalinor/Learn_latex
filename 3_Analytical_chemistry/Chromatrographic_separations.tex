\documentclass{article}
\usepackage[utf8]{inputenc}
\usepackage{bibtex} %Bibliography

\title{Chromatrographic Separations}
\autor{Dervalinor}
\date{Mar 2023}

\begin{document}

\maketitle

\section{Introduction}

Chromatographic separation techniques are multi-stage separation methods in
which the components of a sample are distributed between 2 phases, one of which
is stationary, while the other is mobile. The stationary phase may be a solid
or a liquid supported on a solid or a gel. The stationary phase may be packed
in a column, spread as a layer, or distributed as a film, etc. The mobile phase
may be gaseous or liquid or supercritical fluid. The separation may be based on
adsorption, mass distribution (partition), ion exchange, etc., or may be based
on differences in the physico-chemical properties of the molecules such as
size, mass, volume, etc \cite{convo}.

\begin{thebibliography}
  \bibitem{convo}
  \textit{CHROMATOGRAPHIC SEPARATION TECHNIQUES}.
  Retrieved from \url{https://www.drugfuture.com/Pharmacopoeia/EP7/DATA/20246E.PDF} on March 19, 2023.
\end{thebibliography}

\end{document}
