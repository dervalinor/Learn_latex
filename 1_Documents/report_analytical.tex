\documentclass{article}
\usepackage{graphicx}
\usepackage{wrapfig}

\begin{document}

\begin{wrapfigure}{l}{0.2\textwidth}
    \includegraphics[width=0.2\textwidth]{university-logo.png}
\end{wrapfigure}
\textbf{University Name}

Lorem ipsum dolor sit amet, consectetur adipiscing elit. Nulla ornare, metus a bibendum hendrerit, turpis mi tincidunt ipsum, vel malesuada justo magna a velit.

\begin{wrapfigure}{r}{0.2\textwidth}
    \includegraphics[width=0.2\textwidth]{chemistry-department-logo.png}
\end{wrapfigure}
\textbf{Department of Chemistry}

In hac habitasse platea dictumst. Maecenas sed nisi sit amet libero rhoncus tincidunt in quis nulla. Nullam vel lorem quis enim pharetra rutrum. 

\section{Resultados y discusión}

\subsection{Punto de fusión}

Se determinó el punto de fusión de la muestra problema número 4 utilizando un fusiometro con capilar. Se observó que el compuesto comenzó a derretirse a aproximadamente 285 \degree C y terminó en estado líquido a 289.6 \degree C, con una coloración marrón-rojiza.

\subsection{Concentración de la solución}

Se pesó 0.824g de la muestra problema y se aforó a 50mL, obteniendo una solución de una concentración de 0.0999M. Se realizó la medición del índice de refracción utilizando un refractómetro, obteniendo un valor de 1.345.

\subsection{Espectro UV-VIS}

Se realizó un espectro UV-VIS del compuesto, observando que tenía un máximo de absorción en 244nm con una absorbancia de 3.9995.

\subsection{Polarimetría}

Se midió el ángulo de giro de la luz polarizada de la solución por medio del polarímetro, obteniendo un resultado de 137.9 \degree. Con este valor se calculó la rotación específica utilizando la ecuación $\beta = {\alpha}_D \times b \times c$, donde b es longitud del porta-muestras (dm), c concentración en g/ml, $\beta$ el ángulo de giro de luz polarizada y ${\alpha}_D$ la rotación específica. Con b = 2.5dm y c = 16.48g/ml, se obtiene que ${\alpha}_D = 3.347087379$. Por lo tanto, la rotación específica para la muestra problema número 4 es de 3.347087379.

\subsection{Espectro IR}

Se obtuvo el espectro IR de la muestra problema número 4 utilizando un espectrofotómetro IR. Se observaron los siguientes picos de bandas de absorción:

\end{document}
